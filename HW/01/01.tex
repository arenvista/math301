\documentclass[11pt]{article}
\usepackage{amsmath,amsthm,amscd,amssymb,mathrsfs,tikz}
\usepackage{latexsym,epsf,epsfig}
\newcommand{\ds}{\displaystyle}
\newcommand{\sU}{\mathscr U}
\usepackage[left=1.2in, right=1.2in, top=1in, bottom=1in]{geometry}
%The above set the parameters adjust the margins. There are many ways to do this, and frankly, what is above is not the best. However, it will work the time being.
\usepackage{sectsty}


\usepackage{booktabs, multirow} % for borders and merged ranges
\usepackage{soul}% for underlines
\usepackage{xcolor,colortbl} % for cell colors
\usepackage{changepage,threeparttable} % for wide tables

\usepackage{hyperref}
\theoremstyle{plain}
\newtheorem{theorem}{Theorem}

\usepackage{graphicx}

% Margins
\topmargin=-0.45in
\evensidemargin=0in
\oddsidemargin=0in
\textwidth=6.5in
\textheight=9.0in
\headsep=0.25in

\title{ MATH 301: Homework 1}
\author{ Aren Vista }
\date{\today}

\begin{document}
\maketitle	
\pagebreak

\section*{Question 1}
\begin{proof} Show that $A \cap (B \cup C) = (A \cap B) \cup (A \cap C)$.
    \subsection*{Case 1: $A \cap (B \cup C) \subset (A \cap B) \cup (A \cap C)$}
    Consider an arbtrary element x: \[
        x \in A \cap (B \cup C) \implies x \in A \land x \in (B \cup C)
    \]
    Logically: \[
        x \in A \land x \in (B \cap C) \implies x \in A \land (x \in B \lor x \in C)
    \]
    By distributive property \[
        x \in A \land (x \in B \lor x \in C) \equiv (x \in A \land x \in C) \lor (x \in A \land x \in B)
    \]
    Redefinig as logical statment to set inclusion yields: \[
        (x \in A \land x \in C) \lor (x \in A \land x \in B) \implies x \in (A \cap C) \cup (A \cap B)
    \]
    As x is an arbtrary element Case 1: $A \cap (B \cup C) \subset (A \cap B) \cup (A \cap C)$ holds

    \subsection*{Case 2: $(A \cap B) \cup (A \cap C) \subset A \cap (B \cup C)$}
    Notice all implications can be done in reverse as: \\
    Consider an arbtrary element x: \[
        x \in A \cap (B \cup C) \leftrightarrow x \in A \land x \in (B \cup C)
    \]
    Logically: \[
        x \in A \land x \in (B \cap C) \leftrightarrow x \in A \land (x \in B \lor x \in C)
    \]
    By distributive property \[
        x \in A \land (x \in B \lor x \in C) \equiv (x \in A \land x \in C) \lor (x \in A \land x \in B)
    \]
    Redefinig as logical statment to set inclusion yields: \[
        (x \in A \land x \in C) \lor (x \in A \land x \in B) \leftrightarrow x \in (A \cap C) \cup (A \cap B)
    \]
    Thus as x is an arbtrary element Case 2: $(A \cap B) \cup (A \cap C) \subset A \cap (B \cup C)$ holds
    \subsection*{Conclusion}
    As Case 1,2 holds the two sets must be equal
\end{proof}

\section*{Question 2}
\begin{proof}Show that is $f: A \rightarrow B$ is surjective and $H \subseteq B$ then $f(f^{-1}(H)) = H$. \\
    Suppose $f: A \rightarrow B$ is surjective and $H \subseteq B$ \\
    Observe $f: A \rightarrow B$\[
        f = \{f(a) : a \in A \} \implies f(a) = b, b \in B
    \]
    As $f$ is surjective\[
        \forall ~b \in B, \exists ~a \in A \ni f(a) = b
    \]
    Observe \[
        H \subseteq B \implies \forall ~h \in H, h \in B
    \]
    Thus for some $X \subset A$ \[
        f(X) = {f(x) | x \in X} = H
    \]
    By definition of pre-image \[
        f^{-1}(H) = {x | f(x) \in H} = X
    \]
    Thus \[
        f(f^{-1}(H)) = f(X) = H
    \]
\end{proof}

\begin{proof}Show that is $f: A \rightarrow B$ then $f(f^{-1}(H)) \subseteq H$ \\
    Suppose $f: A \rightarrow B$
    Observe $f: A \rightarrow B$\[
        f = \{f(a) : a \in A \} \implies f(a) = b, b \in B
    \]
    As $f$ is not surjective\[
        \exists ~b \in B \ni \forall ~a \in A, f(a) \neq b
    \]
    Consider some $X \subset A$ \[
        f(X) = {f(x) \in H | x \in X}
    \]
    As $f$ is not surjective $f(X)$ may not equal $H$ \\
    Thus there may exists an $h \in H$ s.t. $f^{-1}(h)$ is not defined. \\
    Therefore, $f(f^{-1}(H)) \subseteq H$ 
\end{proof}

\section*{Question 3}
Theorem 1.1.14: 
\begin{itemize} 
    \item[] Let $f: A \rightarrow B$ be a function
    \item[] Let $g: B \rightarrow C$ be function
    \item[] Let $H \subset C$
    \item[] Prove $(gof)^{-1}(H) = f^{-1}(g^{-1}(H))$
\end{itemize}

\begin{proof} $ = f^{-1}(g^{-1}(H))$
    Let $f: A \rightarrow B$ be a function \[
        f = \{f(a) \in B : a \in A\}
    \]
    Let $g: B \rightarrow C$ be functions\[
        g = \{g(b) \in C : b \in B\}
    \]
    Let $H \subset C$ \[
        \forall ~h \in H, h \in C
    \]
    \subsection*{Case 1: $(gof)^{-1}(H) \subset f^{-1}(g^{-1}(H))$}
    Let $x$ be an arbitrary element $x \in (gof)^{-1}(H)$
    \begin{align*}
        x \in (g \circ f)^{-1}(H) & \iff (g \circ f)(x) \in H 
            && \text{(By definition of inverse image)} \\
        & \iff g(f(x)) \in H
            && \text{(By definition of composition)} \\
        & \iff f(x) \in g^{-1}(H) 
            && \text{(By definition of inverse image for } g \text{)} \\
        & \iff x \in f^{-1}(g^{-1}(H)) 
            && \text{(By definition of inverse image for } f \text{)}
    \end{align*}
    \subsection*{Case 2: $f^{-1}(g^{-1}(H)) \subset (gof)^{-1}(H) $}
    Let $x$ be an arbitrary element $(gof)^{-1}(H)$ and (traverse statements in reverse order)
    \begin{align*}
        x \in (g \circ f)^{-1}(H) & \iff (g \circ f)(x) \in H 
            && \text{(By definition of inverse image)} \\
        & \iff g(f(x)) \in H
            && \text{(By definition of composition)} \\
        & \iff f(x) \in g^{-1}(H) 
            && \text{(By definition of inverse image for } g \text{)} \\
        & \iff x \in f^{-1}(g^{-1}(H)) 
            && \text{(By definition of inverse image for } f \text{)}
    \end{align*}
    \subsection*{Conclusion}
    Thus, as Case 1,2 holds $(gof)^{-1}(H) = f^{-1}(g^{-1}(H))$ holds.
\end{proof}
\section*{Question 4}
Prove that $n^3+5n$ is divisible by 6 for all $n \in \mathbb{N}$; skipping too easy
\section*{Question 5}
Conjecture a formula for the sum of the first $n$ odd natural numbers $1+3+...+(2n-1)$ Prove via induction. \[
    P(n) = \sum^n_{i=1}{(2i-1)} = n^2
\]
\begin{proof}
    Let $P(n) = \sum^n_{i=1}{(2i-1)} = n^2$ \\
    Observe\[ 
        P(1) = 1 = 1^2
    \]
    Thus $P(1)$ holds \\
    Suppose $P(k), k \in \mathbb{N}$ holds. Thus, \[
        P(k) = \sum^k_{i=1}{(2i-1)} = k^2
    \]
    Add $2(k+1)-1 = 2k+1$ to both sides of the eqution
    \begin{align*}
        \sum^k_{i=1}{(2i-1)} + (2k+1) = k^2 + (2k+1) \\
        \sum^{k+1}_{i=1}{(2i-1)} = k^2+2k+1
    \end{align*}
    Observe \[
        P(k+1): \sum^{k+1}_{i=1}{(2i-1)} = (k+1)^2 = k^2+2k+1
    \]
    Thus $P(k+1)$ holds. \\
    By PMI $P(n)$ holds $\forall ~n \in \mathbb{N}$
\end{proof}

\end{document}
