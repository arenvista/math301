\documentclass[11pt]{article}
\usepackage{amsmath,amsthm,amscd,amssymb,mathrsfs,tikz}
\usepackage{latexsym,epsf,epsfig}
\newcommand{\ds}{\displaystyle}
\newcommand{\sU}{\mathscr U}
\usepackage[left=1.2in, right=1.2in, top=1in, bottom=1in]{geometry}
%The above set the parameters adjust the margins. There are many ways to do this, and frankly, what is above is not the best. However, it will work the time being.
\usepackage{sectsty}


\usepackage{booktabs, multirow} % for borders and merged ranges
\usepackage{soul}% for underlines
\usepackage{xcolor,colortbl} % for cell colors
\usepackage{changepage,threeparttable} % for wide tables

\usepackage{hyperref}
\theoremstyle{plain}
\newtheorem{theorem}{Theorem}

\usepackage{graphicx}

% Margins
\topmargin=-0.45in
\evensidemargin=0in
\oddsidemargin=0in
\textwidth=6.5in
\textheight=9.0in
\headsep=0.25in

\title{ MATH 301: Homework 2}
\author{ Aren Vista }
\date{\today}

\begin{document}
\maketitle	
\pagebreak

\section*{Question 1}
Prove Theorem 1.3.4 (b) \\
If a set $A$ with $m \in \mathbb{N}$ elements and $C \subseteq A$ is a set with one element
then $A-C$ is a set with $m-1$ elements
\begin{proof}
    Suppose a set $A$ with $m \in \mathbb{N}$ elements \[
        A = \{a_1, a_2, ..., a_{m-1}, a_m\} \implies |A| = m
    \]
    Suppose $C \subseteq A$ is a set with one element \[
        C = \{a\}, a \in A
    \]
    Observe a is an arbtrary element of A \\
    WLOG let $a=a_m$ s.t. \[
        C = \{a_m\}
    \]
    Thus \[
        A-C = \{a_1, a_2, ..., a_{m-1}\} \implies |A-C| = m-1
    \]
\end{proof}
\section*{Question 2}
Prove Theorem 1.3.4 (c) \\
If $C$ is an infinite set and $B$ is a finite set \\
Then $C-B$ is an infinite set \\
Recall an infinite set is a set which is not finite (defn.) \\
\begin{proof}
    Suppose $C$ is an infinite set \[
        C = \{c_1, c_2, ...\}
    \]
    Suppose $B$ is a finite set \[
        B = \{b_1, b_2, ..., b_n\}, n \in \mathbb{N}
    \]
    Suppose for the sake of contradiction $C-B$ is finite s.t \[
        C-B = \{x_1, x_2, ..., x_m\} m \in \mathbb{N}
    \]
    Observe at most $C \cap B$ must be finite (n number of elements) by definition of set intersection \[
        C \cap B = \{b_1, b_2, ..., b_n\}
    \]
    By rearangement of the principle of exclusion \[
        C = C-B \cup (C \cap B)
    \]
    If $C-B$ is finite and  $C \cap B$ is finite then $C$ must be finite. This is a logical inconsistency. \\
    Thus, C-B must be infinite.
\end{proof}

\section*{Question 3}
Prove that if $x$ is a rational number and $y$ is an irrational number, then $x+y$ is an irrational number. \\
If in addition $x \neq 0$ then show that $xy$ is an irrational number.
\begin{proof}
    Suppose $x$ is a rational number and $y$ is an irrational number \\
    Suppose for the sake of contradiction $x+y$ is a rational number. \\
    Then given $m,n \in {Z}$ \[
        x+y = \frac{m}{n}
    \]
    By defn. $x,k,p \in \mathbb{Q} \implies x = \frac{k}{p}$ \\
    Thus
    \begin{align*}
        \frac{k}{p} + y = \frac{m}{n} \\
        y = \frac{m}{n} - \frac{k}{p} \\
        y = \frac{mp}{np} - \frac{kn}{pn} \\
        y = \frac{mp-kn}{np} \\
    \end{align*}
    By set inclusion of integers, $mp-kn, np \in \mathbb {Z}$. This would imply $y$ is rational. \\
    This is a logical inconsistency, thus $x+y$ is irrational.
\end{proof}

\section*{Question 4}
Modify the proof of Theorem 2.1.4 \\
Show that there does not exist a rational number $t$ so that $t^2 = 3$ \\
\begin{proof}
    Let $t \in \mathbb{Q}$ \\
    Suppose for the sake of contracdiction $t^2 = 3$ \\
    Let $x,y \in \mathbb{Z}$, y defn of $t = \frac{x}{y}$, where $x,y$ are the lowest possible factors of $t$ \\
    Thus
    \begin{align*}
        (\frac{x}{y})^2 = 3 \\
        \frac{x^2}{y^2} = 3 \\
        x^2 = 3y^2 \\
    \end{align*}
    Notice $x^2$ is a multiple of 3 \\
    By properties of prime \[
        3|x^2 \implies 3|x
    \]
    Thus for some $k \in \mathbb{Z}$ \[
        x^2 = (3k)^2 = 3y^2 
    \]
    Thus $3k = y^2$ \\
    This concludes $y|3$ and $x|3$. \\
    Note, we stated $x,y$ are the lowest possible factors of $t$ implying there does not exist any more common factors which could reduced $x,y$ \\
    This is a logical inconsistency, this there can not exist a $t \in \mathbb{Q}$ s.t. $t^2 = 3$
\end{proof}

\section*{Question 5}
Recall the \textbf{Triangle Inequality} \\
If $a,b \in \mathbb{R}$ \\
Then $|a+b| \leq |a| + |b|$ \\
Prove this inequality holds iff $ab \geq 0$ \\
Statement P:$|a+b| \leq |a| + |b| \iff ab \geq 0$
\subsubsection*{Case 1: $a$ and $b$ have the same sign}
$a$ and $b$ have the same sign $\implies ab \geq 0$:
Observe
\begin{itemize}
    \item[] $a$ and $b$ are positive \\ $\implies |a+b| \equiv a+b \iff a+b \leq |a|+|b| \iff |a+b| \leq |a| + |b|$
    \item[] $a$ and $b$ are negative \\ $\implies |(-a)+(-b)| = |-(a+b)| \equiv a+b \iff a+b \leq |a|+|b| \iff |a+b| \leq |a| + |b|$
\end{itemize}

\subsubsection*{Case 2: $a=0 \lor b=0$}
$a=0 \lor b=0 \implies ab \geq 0$
Observe
\begin{itemize}
    \item[] If $a=0$ \\ $\implies |0+b| = |b| \equiv b \iff b \leq 0+|b| \iff |a+b| \leq |a| + |b|$
    \item[] If $b=0$ \\ $\implies |0+a| = |a| \equiv a \iff a \leq 0+|a| \iff |a+b| \leq |a| + |b|$
\end{itemize}

\subsection*{Conclusion}
Thus By Case 1,2 $|a+b| \leq |a| + |b| \iff ab \geq 0$ holds

\section*{Question 6}
Show that if $a,b \in \mathbb{R}$ and $a \neq b$ \\
Then there exists $\epsilon$-neighborhoods $U$ of $a$ and $V$ of $b$ such that $U \cap V = \emptyset$

\end{document}
